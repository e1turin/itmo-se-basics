%%%%%%%%%%%%%%%%%%%%%%%%%%%%%%%%% LAB-5 %%%%%%%%%%%%%%%%%%%%%%%%%%%%%%%%%%
%>>>>>>>>>>>>>>>>>>>>>>>>>> ПЕРЕМЕННЫЕ >>>>>>>>>>>>>>>>>>>>>>>>>>>>>>>>>>>
%>>>>> Информация о кафедре
%\newcommand{\year}{2021 г.}  % Год устанавливается автоматически
\newcommand{\city}{Санкт-Петербург}  %  Футер, нижний колонтитул на титульном листе
\newcommand{\university}{Национальный исследовательский университет ИТМО}  % первая строка
\newcommand{\department}{Факультет программной инженерии и компьютерной техники}  % Вторая строка
\newcommand{\major}{Направление системного и прикладного программного обеспечения}  % Треьтя строка
%<<<<< Информация о кафедре

%>>>>> Назание работы
\newcommand{\reporttype}{ОТЧЕТ ПО ЛАБОРАТОРНОЙ РАБОТЕ} % тип работы, (главный заголовок титульного листа)
\newcommand{\lab}{Лабораторная работа}          % вид работы
\newcommand{\labnumber}{№ 3}                    % порядковый номер работы
\newcommand{\subject}{Методы и средства программной инженерии}         % учебный предмет
\newcommand{\labtheme}{Системы сборки приложений и управления зависимостей}  % Тема лабораторной работы
\newcommand{\variant}{324571}                % номер варианта работы

\newcommand{\student}{Тюрин Иван\\
                      Беляков Дмитрий}    % определение ФИО студента
\newcommand{\studygroup}{P32131}                 % определение учебной группы 
\newcommand{\teacher}{Бострикова Д. К.}          % ФИО практика
%<<<<<<<<<<<<<<<<<<<<<<<<<< ПЕРЕМЕННЫЕ <<<<<<<<<<<<<<<<<<<<<<<<<<<<<<<<<<<


%>>>>>>>>>>>>>>>>>>>>>> ПРЕАМБУЛА >>>>>>>>>>>>>>>>>>>>>>>>>
\documentclass{scrreprt}
\usepackage[a4paper, mag=1000, left=3cm, right=1.5cm, top=2cm, bottom=2cm, headsep=0.7cm, footskip=1cm]{geometry} % По ГОСТу: left>=3cm, right=1cm, top=2cm, bottom=2cm,
\linespread{1} % межстройчный интервал по ГОСТу := 1.5
%<<<<< Разметка документа


\usepackage{listings}
\usepackage{underscore}
\usepackage{graphicx}
\usepackage[bookmarks=true]{hyperref}
%>>>>> babel c языковым пакетом НЕ должны быть первым импортируемым пакетом
\usepackage[utf8]{inputenc}
\usepackage[T1,T2A]{fontenc}
\usepackage[russian]{babel}
\usepackage{amsmath,amsthm,amssymb}
\usepackage{mathtext}
\usepackage{float}
\usepackage{svg}
%<<<<<

\hypersetup{
    bookmarks=true,  % show bookmarks bar?
    pdftitle={Software Requirement Specification}, % title
    pdfauthor={Тюрин Иван и Беляков Дмитрий}, % author
    pdfsubject={SRS}, % subject of the document
    pdfkeywords={TeX, LaTeX, graphics, images}, % list of keywords
    colorlinks=true, % false: boxed links; true: colored links
    linkcolor=blue,  % color of internal links
    citecolor=black, % color of links to bibliography
    filecolor=black, % color of file links
    % urlcolor=purple, % color of external links
    linktoc=page     % only page is linked
}%

% \def\myversion{1.0 }

\usepackage{hyperref}
\usepackage{indentfirst}

\newcommand{\FunctionalReq}[6]{
    \item {\Large <<#1>>}\\[1mm]
    \textbf{ID:} \texttt{#2}.\\[1mm]
    \textbf{Приоритет:} \textit{#3}.\\[1mm]
    \textbf{Описание}. #4\\[1mm]
    \textbf{Оценка}. #5\\[1mm]
    \textbf{Зависимости}. \texttt{#6}.\\[1mm]
}

\newcommand{\QualityReq}[3]{
    \item {\Large <<#1>>}\\[1mm]
    \textbf{ID:} \texttt{#2}.\\[1mm]
    \textbf{Описание}. #3\\[1mm]
}

\newcommand{\Scenario}[2]{
    \textbf{Сценарий использования} <<#1>>:
    #2% 
    \vspace{3mm}
}
%<<<<<<<<<<<<<<<<<<<<<< ПРЕАМБУЛА <<<<<<<<<<<<<<<<<<<<<<<<<



%%%%%%%%%%%%%%%%%%% СОДЕРЖИМОЕ ОТЧЕТА %%%%%%%%%%%%%%%%%%%%%
%>>>>>>>>>>>>>>> ''''''''''''''''''''''' >>>>>>>>>>>>>>>>>>
\begin{document}


%>>>>>>>>>>>>>>>> ОПРЕДЕЛЕНИЕ НАЗВАНИЙ >>>>>>>>>>>>>>>>>>>>
% Переоформление некоторых стандартных названий
%\renewcommand{\chaptername}{Лабораторная работа}
\renewcommand{\chaptername}{\lab\ \labnumber} % переименование глав
\def\contentsname{Содержание} % переименование оглавления
%<<<<<<<<<<<<<<<< ОПРЕДЕЛЕНИЕ НАЗВАНИЙ <<<<<<<<<<<<<<<<<<<<
% \setlength{\itemsep}{0pt} % установка расстояния между строчками в списках можно использовать локально внутри списка списке
% \setlength{\parskip}{0pt} % 
% \setlength{\parsep}{0pt}  % 

%>>>>>>>>>>>>>>>>> ТИТУЛЬНАЯ СТРАНИЦА >>>>>>>>>>>>>>>>>>>>>
\begin{flushright}
    \rule{16cm}{5pt}\vskip1cm
    \begin{bfseries}
        \Huge{СПЕЦИФИКАЦИЯ ТРЕБОВАНИЙ\\ К ПРОДУКТУ}\\
        \vspace{1.5cm}
        для\\
        \vspace{1.5cm}
        \projectname\\
        \vspace{1.5cm}
        \LARGE{Версия \myversion}\\
        \vspace{1.5cm}
        Подготовили : \authors\\
        \vspace{1.5cm}
        Приняли : \assignee\\\assigneepost \\
        \vspace{1.5cm}
        %\today\\
        \the\year~г.\\
    \end{bfseries}
\end{flushright}
%<<<<<<<<<<<<<<<<< ТИТУЛЬНАЯ СТРАНИЦА <<<<<<<<<<<<<<<<<<<<<


%>>>>>>>>>>>>>>>>>>>>> СОДЕРЖАНИЕ >>>>>>>>>>>>>>>>>>>>>>>>>
% Содержание
\tableofcontents
%<<<<<<<<<<<<<<<<<<<<< СОДЕРЖАНИЕ <<<<<<<<<<<<<<<<<<<<<<<<<


%%%%%%%%%%%%%%%%%%%%%%% КОД РАБОТЫ %%%%%%%%%%%%%%%%%%%%%%%%
%>>>>>>>>>>>>>>>>>>>'''''''''''''''''>>>>>>>>>>>>>>>>>>>>>
\newpage
\Chapter{\lab\ \labnumber}{\labtheme}{}

\Section{Задание варианта \variant}
\begin{center}
, , ,
\end{center}
Написать сценарий для утилиты Apache Ant, реализующий компиляцию, тестирование и упаковку в jar-архив кода проекта из лабораторной работы №3 по дисциплине "Веб-программирование".

Каждый этап должен быть выделен в отдельный блок сценария; все переменные и константы, используемые в сценарии, должны быть вынесены в отдельный файл параметров; MANIFEST.MF должен содержать информацию о версии и о запускаемом классе.

Cценарий должен реализовывать следующие цели (targets):
\begin{itemize}
    \item \verb|compile| --- компиляция исходных кодов проекта.
    \item \verb|build| --- компиляция исходных кодов проекта и их упаковка в исполняемый jar-архив. Компиляцию исходных кодов реализовать посредством вызова цели compile.
    \item \verb|clean| --- удаление скомпилированных классов проекта и всех временных файлов (если они есть).
    \item \verb|test| --- запуск junit-тестов проекта. Перед запуском тестов необходимо осуществить сборку проекта (цель \verb|build|).
    \item \verb|scp| --- перемещение собранного проекта по scp на выбранный сервер по завершению сборки. Предварительно необходимо выполнить сборку проекта (цель \verb|build|)
    \item \verb|native2ascii| --- преобразование native2ascii для копий файлов локализации (для тестирования сценария все строковые параметры необходимо вынести из классов в файлы локализации).
    \item \verb|doc| --- добавление в MANIFEST.MF MD5 и SHA-1 файлов проекта, а также генерация и добавление в архив javadoc по всем классам проекта.
    \item \verb|diff| --- осуществляет проверку состояния рабочей копии, и, если изменения касаются классов, указанных в файле параметров выполняет commit в репозиторий svn.
    \item \verb|history| --- если проект не удаётся скомпилировать (цель \verb|compile|), загружается предыдущая версия из репозитория svn. Операция повторяется до тех пор, пока проект не удастся собрать, либо не будет получена самая первая ревизия из репозитория. Если такая ревизия найдена, то формируется файл, содержащий результат операции \verb|diff| для всех файлов, измёненных в ревизии, следующей непосредственно за последней работающей.
    \item \verb|report| --- в случае успешного прохождения тестов сохраняет отчет junit в формате xml, добавляет его в репозиторий svn и выполняет commit.
\end{itemize}
\begin{center}
' ' '
\end{center}


\Section{Выполнение задания}
\Subsection{Листинг программы}
Разработанный конфигурационный файл для ant представлен на листинге \ref{lst:ant-build}.

\lstinputlisting[caption={\texttt{build.xml} --- конфигурационыый файл для ant},label={lst:ant-build},language=xml]{./res/ant-build.txt}

\Subsection{Листинг тестов}
Так же для работы скрипта были добавлены тесты с использованием бублиотеки Junit 5. Тестовый класс можно видеть на листинге \ref{lst:junit}.
\lstinputlisting[caption={Тестовый класс разработанный с использованием Junit 5 и параметризированных тестов},label={lst:junit},language=java]{./res/tests.txt}

\Section{Вывод}
Современные системы сборки нереально сложные системы и гораздо более удобные в сравнении с императивным конфигурированием с помощью XML на ant. Gradle теперь выглядит как супер удобный инструмент без богомерзкого XML. Тестирование приложений --- не простая задача, и здорово, что есть современные иструменты позволяющие автоматизировать эту деятельность.
Всё замечательно, лаба просто супер, спасибо.
\newpage
%<<<<<<<<<<<<<<<<<<<<<< КОД РАБОТЫ <<<<<<<<<<<<<<<<<<<<<<<<

%>>>>>>>>>>>>>>>> СПИСОК ЛИТЕРАТУРЫ >>>>>>>>>>>>>>>>>>>>>>>
%\include{biblist}  % Для соответсвия гост, придется доработать. Нужен файл .bib
%<<<<<<<<<<<<<<<<<<<< СПИСОК ЛИТЕРАТУРЫ <<<<<<<<<<<<<<<<<<<

\end{document}
%<<<<<<<<<<<<<<<< ,,,,,,,,,,,,,,,,,,,,,,, <<<<<<<<<<<<<<<<<
%<<<<<<<<<<<<<<<<<<< СОДЕРЖИМОЕ ОТЧЕТА <<<<<<<<<<<<<<<<<<<<
