В данном документе далее следует 3 главы: <<Описание системы>> и <<Требования к реализуемой системе>>, в каждой из которых в отдельных секциях определены описываемые аспекты разрабатываемой системы. В конце документа находится Приложение, в котором указаны выводы по работе.

В главе <<Описание системы>> содержатся аспекты касающиеся условий использования системы. Эти аспекты принимаются во внимание при выдвижении технических требований в главе <<Требования к реализуемой системе>>. В этой главе описаны конечные пользователи системы, условия эксплеатации сервисов и приведены оценки возможных рисков с методами по противодействию им.

В главе <<Требования к реализуемой системе>> описаны все функциональные и нефункциональные требования, внешние интерфейсы и сценарии использования системы. 

% \textit{[ЗАПОЛНИМ ПОЗЖЕ,  Т.К. ТУТ ДОЛЖНЫ БЫТЬ ОПИСАНО СЛЕДУЮЩЕГО ДАЛЕЕ СОДЕРЖИМОЕ ДОКУМЕНТА]}

%Сайт предназначен для информирования населения Амурской области о последних событиях в их населенных пунктов и целого региона, прогнозах погоды, телепрограммах, частных объявлениях и вакансиях Благовещенска, киноафишах и расписаниях транспорта.

%АМУР.Инфо — свежие новости Благовещенска, Амурской области, Дальнего Востока и России. Статьи, интервью, видео. Узнай первым с Амур.инфо.

% Своевременное и централизованное %%тут что-то про свойства информации можно сказать%%
% Информирование населения есть важная составляющая современного общества. Поэтому необходимо разработать сайт для регионального Информационного Агентства «Амур.инфо» по Амурской области. Главная цель сайта --- это позволить централизованно информировать жителей Амурской области об актуальных и важных событиях в их населенном пункте и в целом регионе. Дополнительной задачей сайта является предоставление эргономичного доступа его посетителям к новостным материалам релевантным для них.
%, это можно осуществить в виде тематических подборок для разных типов посетителей и поиска новостей по определенным рубрикам и указаному промежутку времени.

%Информация обладает некоторыми свойствами: достоверность, объективность, полнота, актуальность, понятность, доступность, релевантность, эргономичность.

% The remainder of this document includes three chapters and appendixes. The second one provides an 
% overview of the system functionality and system interaction with other systems. This chapter also 
% introduces different types of stakeholders and their interaction with the system. Further, the chapter also 
% mentions the system constraints and assumptions about the product.
% The third chapter provides the requirements specification in detailed terms and a description of the 
% different system interfaces. Different specification techniques are used in order to specify the 
% requirements more precisely for different audiences.
% The fourth chapter deals with the prioritization of the requirements. It includes a motivation for the 
% chosen prioritization methods and discusses why other alternatives were not chosen. 
% The Appendixes in the end of the document include the all results of the requirement prioritization and a 
% release plan based on them

%%%%
% This subsection should
% а) Describe what the rest of the SRS contains
% Ь) Explain how the SRS is organized