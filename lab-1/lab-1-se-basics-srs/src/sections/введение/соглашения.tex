%Introduction \ Document conventions 
%?Intended Audience and Reading Suggestions
В этой секции представлены значения терминов, используемых в дальнейшем содержимом документа.\\[2mm]

\noindent
\textit{Автор} --- человек, имеющий возможность создавать данные новостных материалов и редактировать только созданные им самим или другими людьми из его рабочей команды.
\\[2mm]
\textit{Администратор} --- человек, управляющий сайтом, обладающий возможностью редактировать любые данные на сайте и выполнять действия для поддержания работы сайта.
\\[2mm]
\textit{Веб-портал} --- страницы веб-сайта, с которыми происходит взаимодействие посетителей, администраторов, редакторов.
\\[2mm]
\textit{Десктопное устройство} --- обозначение, используемое для обозначения стационарных компьютеров, ноутбуков, нетбуков, моноблоков и ноутбуков-трансформеров. Доступ к сайту осуществляется с помощью веб-браузера.\\[2mm]
\textit{Мобильное устройство} --- обозначение, используемое для обозначения мобильных телефонов, смартфонов, камерофонов, коммуникаторов и планшетов.\\[2mm]
\textit{Посетитель} или \textit{гость} --- человек, зашедший на веб-страницу сайта по прямому назначению: для просмотра новостных материалов. 
\\[2mm]
\textit{Редактор} --- человек, имеющий возможность создавать и редактировать данные только новостных материалов.
%%%%%%% По сути это должно заполняться после выполнения основной части %%%%%%%%%%%

%Здесь мы описываем все непонятные технические слова или термины которые встречаются в SRS. Заметьте, что описание непонятного слова не может содержать другое непонятное слово. Старайтесь расписать как можно подробнее термин который Вы используете простым и понятным всем языком. Не экономьте на этой секции потому, что чем больше вы распишете непонятных вещей, тем проще будет потом проектировать.