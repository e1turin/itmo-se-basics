Всех пользователей веб-портала можно разделить на 3 категории: гости, администраторы и редакторы.

Гости в свою очередь представляют группу людей, которых можно разделить еще на несколько категорий: \begin{enumerate}
    \item люди, интересующиеся исключительно важными новстями в регионе и желающие сразу видеть их на главной странице сайта,
    \item люди, интересующиеся разными категориями новостей и пользующиеся дополнительным поиском по всей базе хранимых новостей.   
\end{enumerate}

Администраторы представляют группу людей, обладающими знанием о внутреннем (без углубления в технические особенности) устройстве системы и пользующиеся всеми возможностями этой системы.

Редакторы представляют группу людей, которые обладают знанием о процессе создания и публикации новостных материалов на веб-портале. Они могут использовать поиск со сложными параметрами по всем имеющимся в базе новостям для выполнения рабочих задач.
% \\
% \textit{[В ЭТОМ ПОДРАЗДЕЛЕ SRS ДОЛЖНЫ БЫТЬ ОПИСАНЫ 
% ОБЩИЕ ХАРАКТЕРИСТИКИ ПРЕДПОЛАГАЕМЫХ ПОЛЬЗОВАТЕЛЕЙ ПРОДУКТА,
%     ВКЛЮЧАЯ УРОВЕНЬ ОБРАЗОВАНИЯ, 
%     ОПЫТ И ТЕХНИЧЕСКИЕ ЗНАНИЯ.
% ЕГО НЕ СЛЕДУЕТ ИСПОЛЬЗОВАТЬ ДЛЯ УКАЗАНИЯ КОНКРЕТНЫХ ТРЕБОВАНИЙ,
% А СКОРЕЕ НУЖНО УКАЗЫВАТЬ ПРИЧИНЫ, ПО КОТОРЫМ ОПРЕДЕЛЕННЫ ЭТИ ТРЕБОВАНИЯ, УКАЗАННЫЕ ДАЛЕЕ В РАЗДЕЛ З SRS]}

%%%%
% This subsection of the SRS should describe those general
% characteristics of the intended users of the product 
% including educational level, experience, and technical 
% expertise. It should not be used to state specific requirements 
% but rather should provide the reasons why certain specific 
% requirements are later specified in section З of the SRS.