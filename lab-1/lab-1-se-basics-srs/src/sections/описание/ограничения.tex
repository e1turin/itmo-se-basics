При разработке системы необходимо учитывать имеющиеся ограничения в выборе технологий. Среди ограничений выделяются
\begin{enumerate}
    \item ограниченный бюджет рассчитанный на создание работоспособной и доступной к развитию версии системы, %--> Использование популярной и дешевой технологии для разработки сайта
    
    \item трудная экономическая обстановка с компьютерным оборудованием в стране, выраженная в дороговизне нового оборудования и большом объеме вычислительных систем бывших в использовании, % --> использование разработка системы с низким потреблением ресурсов эвм и легкой переносимостью между хостами
    
    \item устройства, с помощью которых происходит доступ к сайту, у конечных пользователей может не обладать большой производительностью,
    
    % \item \textit{И Т. Д... (мб что-то про hsts --- мб лучше в 2.5)}
\end{enumerate}

%%%%
% This subsection of the SRS should provide а general description of апу other items that will limit the developer's options. These include
%    а) Regulatory policies
%    Ь) Hardware limitations (for example, signal timing requirements)
%    с) Interfaces to other applications
%    d) Parallel operation
%    е) Audit functions
%    f) Control functions
%    д) Higher-order language requirements
%    h) Signal handshake protocols (for ехатр1е, XON-XOFF, ACk-NACk)
%    i) Reliability requirements
%    ј) Criticality of the application
%    К) Safety and security considerations
    
%1. Ограничение на размер загружаемых файлов - файлы, загружаемые на сайт, должны быть ограничены определенным размером, чтобы не превышать допустимые значения для сервера и не замедлять процесс загрузки страниц.

%2. Ограничение на количество запросов - для предотвращения DDoS-атак и других типов атак на сайт, можно установить ограничение на количество запросов, поступающих с одного IP-адреса.

%3.Ограничение на доступ к определенным разделам сайта - для управления доступом к конфиденциальной информации или функциональности, можно установить ограничения на доступ к определенным разделам сайта для разных категорий пользователей.

%4.Ограничение на время сессии - для безопасности пользователей и сохранения ресурсов сервера, можно установить ограничение на время активной сессии пользователя на сайте.

%5.Ограничение на использование ресурсов сервера - для управления использованием ресурсов сервера, можно установить ограничения на количество запросов, обрабатываемых за определенный период времени, или ограничить количество пользователей, имеющих доступ к сайту одновременно.

%6.Ограничение на ввод данных - для обеспечения безопасности и предотвращения ошибок ввода данных, можно установить ограничения на формат и тип данных, которые могут быть введены пользователем.

%7.Ограничение на использование сторонних сервисов - для обеспечения безопасности и сохранения конфиденциальности информации, можно установить ограничения на использование сторонних сервисов, таких как платежные системы или социальные сети.

%8.Ограничение на использование определенных браузеров - для обеспечения совместимости и удобства использования, можно установить ограничения на использование определенных браузеров или версий браузеров.

%9.Ограничение на доступ к API сайта - для управления доступом к API сайта, можно установить ограничения на количество запросов, которые могут быть выполнены за определенный период времени, или ограничить доступ только для определенных IP-адресов.

%10.Ограничение на использование определенных функций сайта - для управления использованием определенных функций сайта, можно установить ограничения на количество раз, которое пользователь может использовать эти функции за определенный период времени.
