В этой секции описаны требования к подходу к реализации системы, методологии разработки, дизайну архитектуры системы.

\begin{enumerate}
    \QualityReq{Развитие системы}{QR13}{Система должна позволять дальнейшее развитие и расщирение функционала.}
    \QualityReq{Отсутствие привязки к платформе}{QR14}{Система не должна быть привязана к определенной операционной системе.}
    \QualityReq{Одинаковая функциональность}{QR15}{Клиентское приложение должно иметь одинаковую функциональность на любой возможной платформе пользователя.}
    \QualityReq{Адаптивный интерфейс}{QR16}{Пользовательский интерфейс гостей должен быть адаптирован как под десктопные, так и под мобильные устройства.}
    \QualityReq{Лицензирование}{QR17}{Система должна состоять из компонентов, лицензия которых допускает коммерческое использование, конечный выбор должен быть оговорен с заказчиком продукта.}
\end{enumerate}


% \textit{[5.3.5 in IEEE] This should specify design constraints that сап be imposed by other standards, hardware limitations, etc.}

% \subsection{Standart compliance}
% This subsection should specify the requirements derived from existing standards or regulations. They
% тау include

% а) Report format
% Ь) Data naming
% с) Accounting procedures
% d) Audit tracing

% For ехатр1е, this could specify the requirement for software to trace processing activity. Such traces ате
% needed for some applications to meet minimum regulatory ог fnancial standards. Ап audit trace requirement
% тау, for ехатр1е, state that a11 changes to а payroll database must be recorded in а trace fle with before and
% after values.