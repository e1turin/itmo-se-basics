\begin{enumerate}

\FunctionalReq{Авторизация}{FR1}{высокий}{Система должна предоставлять возможность администратору авторизироваться в системе.}{20 часов (реализация функциональности, добавление в пользовательский интерфейс, интеграция с базой данных).}{-}
% (для редактора тоже +20, суммарно 40)
\Scenario{Администратор авторизуется в системе}{
    \begin{itemize}
        \item Администратор хочет авторизоваться в системе;
        \item Когда администратор выполняет вход в систему с помощью учетной записи <<Администратора>>,
        \item Тогда администратор должен быть авторизован как <<Администратор>>.
        % \item Администратор заходит на страницу с формой авторизации;
        % \item Вводит авторизационные данные: логин и пароль, и нажимает кнопку <<Войти>>;
        % \item Данные оказались верны и пользователь попадает на страницу управления сайтом и считается авторизованным, иначе пользователю предлагается повторно ввести данные.
    \end{itemize}
}

\FunctionalReq{Управление пользователями}{FR2}{высокий}{Система должна предоставлять возможность администратору добавлять, редактировать и удалять пользователей в системе.}{35 часов (реализация функциональности, добавление в пользовательский интерфейс, интеграция с базой данных).}{FR1}
%(целой недели как-то много, но задача трудоемкая)
\Scenario{Добавить учетную запись пользователя}{
    \begin{itemize}
        \item Данный администратор авторизован в системе;
        \item Когда администратор хочет добавить нового пользователя в систему,
        \item Тогда система должна добавить нового пользователя в базу данных.
    \end{itemize}
}
\Scenario{Редактировать учетную запись пользователя}{
    \begin{itemize}
        \item Данный администратор авторизован в системе;
        \item Изменяемая учетная запись пользователя должна существовать в системе;
        \item Когда администратор хочет редактировать учетную запись в системе,
        \item Тогда система должна изменить данные этого пользователя в базе данных в соответствии.
    \end{itemize}
}
\Scenario{Удаление учетной записи пользователя}{
    \begin{itemize}
        \item Данный администратор авторизован в системе;
        \item Удаляемая учетная запись пользователя должна существовать в системе;
        \item Когда администратор хочет удалить учетную запись пользователя в системе,
        \item Тогда система должна изменить данные этого пользователя в базе данных.
    \end{itemize}
}
\FunctionalReq{Резервное копирование данных}{FR3}{высокий}{Система должна предоставлять возможность администратору делать резервное копирование всех данных.}{30 часов (реализация функциональности, добавление в пользовательский интерфейс, интеграция с базой данных).}{FR1}
%(одна кнопочка + ui для контроля и простой дамп базы + время на тестирование, т.к. суперважно)
\Scenario{Резервное копирование данных}{
    \begin{itemize}
        \item Данный администратор авторизован в системе;
        \item Когда администратор хочет выполнить резервное копирование системы данных системы,
        \item Тогда система должна сделать резервное копирование базы данных.
    \end{itemize}
}
\FunctionalReq{Управление новостным материалом}{FR4}{высокий}{Система должна предоставлять возможность администратору создавать новости, редактировать, скрывать и удалять любые, находящиеся в системе.}{40 часов (реализация функциональности, добавление в пользовательский интерфейс, интеграция с базой данных).}{FR1}
\Scenario{Создание нового новостного материала}{
    \begin{itemize}
        \item Данный администратор авторизован в системе;
        \item Когда администратор хочет добавить новый новостной материал,
        \item Тогда система должна добавить данный материал в базу данных.
    \end{itemize}
}
\Scenario{Редактирование новостного материала}{
    \begin{itemize}
        \item Данный администратор авторизован в системе;
        \item Редактируемый материал находится в базе данных;
        \item Когда администратор хочет изменить новостной материал,
        \item Тогда система должна отредактировать данный материал в базе данных.
    \end{itemize}
}
\Scenario{Скрытие новостного материала}{
    \begin{itemize}
        \item Данный администратор авторизован в системе;
        \item Скрываемый материал находится в базе данных;
        \item Когда администратор хочет скрыть новостной материал,
        \item Тогда система должна пометить данный материал в базе данных как скрытый.
    \end{itemize}
}
\Scenario{Удаление новостного материала}{
    \begin{itemize}
        \item Данный администратор авторизован в системе;
        \item Удаляемый материал находится в базе данных;
        \item Когда администратор хочет удалить новостной материал,
        \item Тогда система должна удалить данный материал из базе данных.
    \end{itemize}
}

\FunctionalReq{Поиск с фильтрацией по всем материалам}{FR5}{высокий}{Система должна предоставлять возможность администратору искать материалы среди всего контента, расположенного в системе}{40 часов (реализация функциональности, добавление в пользовательский интерфейс, интеграция с базой данных).}{FR1}
\Scenario{Поиск с фильтрацией новостного материала}{
    \begin{itemize}
        \item Данный администратор авторизован в системе;
        \item Администратор вводит фильтр для поиска;
        \item Когда администратор хочет показать отфильтрованные новостные материалы,
        \item Тогда система должна предоставить данные материалы из базы данных администратору.
    \end{itemize}
}

\FunctionalReq{Управление рекламным контентом}{FR6}{высокий}{Система должна предоставлять возможность администратору создавать, добавлять и удалять контент, связанный с рекламой.}{40 часов (реализация функциональности, добавление в пользовательский интерфейс, интеграция с базой данных).}{FR1}
\Scenario{Создание нового рекламного материала}{
    \begin{itemize}
        \item Данный администратор авторизован в системе;
        \item Когда администратор хочет добавить новый рекламный материал,
        \item Тогда система должна добавить данный материал в базу данных.
    \end{itemize}
}
\Scenario{Редактирование рекламного материала}{
    \begin{itemize}
        \item Данный администратор авторизован в системе;
        \item Редактируемый материал находится в базе данных;
        \item Когда администратор хочет изменить рекламный материал,
        \item Тогда система должна отредактировать данный материал в базе данных.
    \end{itemize}
}
\Scenario{Удаление рекламного материала}{
    \begin{itemize}
        \item Данный администратор авторизован в системе;
        \item Удаляемый материал находится в базе данных;
        \item Когда администратор хочет удалить рекламный материал,
        \item Тогда система должна удалить данный материал из базе данных.
    \end{itemize}
}
\end{enumerate}