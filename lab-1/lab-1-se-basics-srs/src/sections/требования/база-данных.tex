В этой секции представлены требования к СУБД на основе предполагаемых сценариев использования сервисов.

\begin{enumerate}
    \QualityReq{Редактирование данных}{QR7}{База данных должна предоставлять возможность редактировать записи.}
    \QualityReq{Авторство записи}{QR8}{Каждая сущность новостного материала должна иметь ссылку на сущность своего создателя.}
    \QualityReq{Безграничный объём}{QR9}{База данных не должна иметь ограничение на максимальных размер хранимых данных, в частности на исходный код разметки новостных материалов.}
    \QualityReq{Безопасность пользовательских данных}{QR10}{Пароли пользователей должны храниться в базе данных в зашифрованном виде.}
    \QualityReq{Качество СУБД}{QR11}{СУБД должна быть хорошей, плохой быть не должна.}
    \QualityReq{Расширяемость СУБД}{QR12}{База данных должна иметь продуманную архитектуру для дальнейшего раcширения и сохранения хорошей производительности.}
\end{enumerate}

% \textit{[5.3.4 in IEEE] This should specify the 10gical requirements for апу infonnation that is to be placed into а database. This
% тау include:
% \\
% а) TYpes of information used Ьу various functions\\
% Ь) Frequency of use\\
% с) Accessing capabilities\\
% d) Data entities and their relationships\\
% е) Integrity constraints\\
% f) Data retention requirements
% }