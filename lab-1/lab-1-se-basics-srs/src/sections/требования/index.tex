Этот раздел содержит описание всех требований к функциональности и характеристикам системы. 
В нем дается подробное описание системы и всех ее возможностей. Требования определены с учетом условий описанных в предыдущей секции.

\section{Внешние интерфейсы системы}
\include{src/главы/введение/index}

\section{Функциональные характеристики системы}
\include{src/главы/введение/index}

\section{Требования к производительности}
в этой секции представлены технические требования к производительности, определенные на основании ранее представленных зависимостей и тревований.

повседневной нагрузкой будем считать обращение пятой части всех жителей региона амурская область к сервису за сутки. это по приблизительным рассчетам составляет $$700\,000\,/\,5\, /\,24\,/\,60\,\approx 100\;\text{запросов/час}.$$

пиковой нагрузкой на систему считается превышение обычной нагрузки в 2 раза.
%здесь видимо таблицу с атрибутами тогда вводить
%в примере первом там через функцию как у нас
%да зачем?

\begin{enumerate}
    \QualityReq{Повседневная нагрузка}{QR1}{Система должна выдерживать повседневную нагрузку неограниченное время.}
    
    \QualityReq{Время безотказной работы}{QR2}{Система должна находиться непрерывно в работоспособном состоянии не менее 95\% времени в течение года.}
    
    \QualityReq{Максимальное время ответа}{QR3}{Система должна давать давать ответ на запрос не дольше чем за 10 с. в условиях нахождения пользователя в амурской области и скорости интернета не менее 250 кбит/с.}
    
    \QualityReq{Время выбора и сортировки}{QR4}{База данных должна выполнять запрос на выбор и сортировку по определенным критериям за время не больше установленного времени ответа на запрос.}
    
    \QualityReq{Резервное копирование}{QR5}{База данных должна давать возможность делать резервные копии в любое время.}
    
    \QualityReq{Пиковая нагрузка}{QR6}{Система должна выдерживать круглосуточно пиковую нагрузку без значительных просадок в производительности в течение пары дней.}
  %  \item [И ТД...]
\end{enumerate}

% \textit{[5.3.3 in IEEE] This subsection should specify both the static and the dynamic numerical requirements placed оп the soft-
% ware or оп human interaction with the software as а whole. Static numerical requirements тау include:
% \\
% а) The ofterminals to be supported\\
% Ь) Пе number of simultaneous users to be supported\\
% с) Amount and of information to бе handled
% \\
% Static numerica1 requirements аге sometimes identifed under а separate section entit1ed capacity.
% \\
% Dynamic numerical requirements тау include, for example, the numbers of transactions and tasks and the
% amount of data to pr(xessed within certain time periods for both normal and реак workload conditions.\\
% All of these requirements should be stated in measurable terms.
% \\
% For ехатр1е,\\
% ~~~95\% ofthe transactions shall be processed in less than 1 s
% \\
% rather than,\\
% ~~~Ап operator shall пот have го waitfor the transaction го complete.
% \\
% NOTE—Numerical limits applied to опе specifc function ате normally specifed as part of the subparagraph
% description of that function.
% }

\section{Логические требования к СУБД}
В этой секции представлены требования к СУБД на основе предполагаемых сценариев использования сервисов.

\begin{enumerate}
    \QualityReq{Редактирование данных}{QR7}{База данных должна предоставлять возможность редактировать записи.}
    \QualityReq{Авторство записи}{QR8}{Каждая сущность новостного материала должна иметь ссылку на сущность своего создателя.}
    \QualityReq{Безграничный объём}{QR9}{База данных не должна иметь ограничение на максимальных размер хранимых данных, в частности на исходный код разметки новостных материалов.}
    \QualityReq{Безопасность пользовательских данных}{QR10}{Пароли пользователей должны храниться в базе данных в зашифрованном виде.}
    \QualityReq{Качество СУБД}{QR11}{СУБД должна быть хорошей, плохой быть не должна.}
    \QualityReq{Расширяемость СУБД}{QR12}{База данных должна иметь продуманную архитектуру для дальнейшего раcширения и сохранения хорошей производительности.}
\end{enumerate}

% \textit{[5.3.4 in IEEE] This should specify the 10gical requirements for апу infonnation that is to be placed into а database. This
% тау include:
% \\
% а) TYpes of information used Ьу various functions\\
% Ь) Frequency of use\\
% с) Accessing capabilities\\
% d) Data entities and their relationships\\
% е) Integrity constraints\\
% f) Data retention requirements
% }

\section{Требования к дизайну системы}
В этой секции описаны требования к подходу к реализации системы, методологии разработки, дизайну архитектуры системы.

\begin{enumerate}
    \QualityReq{Развитие системы}{QR13}{Система должна позволять дальнейшее развитие и расщирение функционала.}
    \QualityReq{Отсутствие привязки к платформе}{QR14}{Система не должна быть привязана к определенной операционной системе.}
    \QualityReq{Одинаковая функциональность}{QR15}{Клиентское приложение должно иметь одинаковую функциональность на любой возможной платформе пользователя.}
    \QualityReq{Адаптивный интерфейс}{QR16}{Пользовательский интерфейс гостей должен быть адаптирован как под десктопные, так и под мобильные устройства.}
    \QualityReq{Лицензирование}{QR17}{Система должна состоять из компонентов, лицензия которых допускает коммерческое использование, конечный выбор должен быть оговорен с заказчиком продукта.}
\end{enumerate}


% \textit{[5.3.5 in IEEE] This should specify design constraints that сап be imposed by other standards, hardware limitations, etc.}

% \subsection{Standart compliance}
% This subsection should specify the requirements derived from existing standards or regulations. They
% тау include

% а) Report format
% Ь) Data naming
% с) Accounting procedures
% d) Audit tracing

% For ехатр1е, this could specify the requirement for software to trace processing activity. Such traces ате
% needed for some applications to meet minimum regulatory ог fnancial standards. Ап audit trace requirement
% тау, for ехатр1е, state that a11 changes to а payroll database must be recorded in а trace fle with before and
% after values.

% \section{Аспекты системы}
% \include{src/главы/введение/index}


%%%%
% This section of the SRS should contain all the software Шrequirements to а level of detail suffcient to enable
% designers to design а system to satisfy those requirements, and testers to test that the system satisfes those
% requirements. Throughout this section, every stated requirement should Ь-е extemally perceivable by users,
% operators, от other external systems. These requirements should include at а minimum а description of every
% input (stimulus) into the system, every output (response) from the system and all functions performed by the
% system in response to ап input от in support of ап output. As this is often the largest and most important part
% of the SRS, the following principles арр1у:
% 
% а) Specifc requirements should be stated in conformance with all the characteristics described in 4.3 0f this recommended practice.\\
% b) Specifc  requirements should be cross-referenced to earlier documents relate.\\
% с) All requirements should be uniquely identifable.\\
% d) Careful attention should given to organizing the requirements to maximize readability.\\
