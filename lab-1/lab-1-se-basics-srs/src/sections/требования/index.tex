Этот раздел содержит описание всех требований к функциональности и характеристикам системы. 
В нем дается подробное описание системы и всех ее возможностей. Требования определены с учетом условий описанных в предыдущей секции.

\section{Внешние интерфейсы системы}
В этом разделе приводится подробное описание всех интерфейсов ввода и вывода. Все интерфейсы разделены на области действия: аппаратные, программного обеспечения, коммуникационные между копмонентами системы и пользовательские. В этом разделе документа так же представлен прототип пользовательского интерфейса веб-портала.

\subsection{Пользовательские интерфейсы}
\begin{enumerate}
    \item <<Главная страница>>\\
    На данной странице система должна предоставлять пользователю набор последних актуальных новостей, а также прогноз погоды, курс валют и доступ к поиску по новостям.
    \item <<Страница с фильтрами>>\\
    На данной странице система должна предоставлять пользователю список фильтров по новостям, а также найденные новости по заданному фильтру.
    \item <<Страница с новостью>>\\
    На данной странице система должна предоставлять пользователю новостной материал по данной ссылке.
    \item <<Страница авторизации>>\\
    На данной странице система должна предоставлять пользователю форму для авторизации пользователя.
    \item <<Страница с панелью управления>>\\
    На данной странице система должна предоставлять авторизованному пользователю визуальный интерфейс для выполнения дейтвий соответствующих роли пользователя.
\end{enumerate}
%\textit{[USER INTERFACES --- UI]}

\subsection{Аппаратные интерфейсы}
Среди аппаратных интерфейсов можно выделить интерфейс доступа в глобальную сеть. 

На стороне клиента должно присутсвовать соедениене с глобальной сетью Интернет по средствам доступных подключений (сотовая связь, проводное подключение, беспроводная сеть Wi-Fi и пр.) для удобства пользователя. Клиентское приложение может отправлять пакеты только на адрес сервера системы и адреса серверов для получения дополнительного ресурса и получать пакеты от них по сети.
%Это хоть и важный интерфейс, но он не заложен  в разработку же
%Мне казалось, что здесь про интерфейсы, которые мы разрабатываем
%думаю стоит оставить, если что потом уберем. Разрабоатывая систему мы в полне можем столкнуться с таким интефейсом, там всякие CORS тоже ведь сюда, кажется

Серверная часть также должна иметь доступ к сети Интернет и локальной сети по средствам проводного подключения для снижения временных задержек на обмен информацией. Серверное приложение может получать пакеты клиентских запросов с любого адреса и отправлять соответствующие пакеты ответов на те же адреса. Так же серверное приложение может отправлять запросы на адрес СУБД, если она физически находится на другом сервере подключенном к сети.

% Поскольку система функционирует на уровне операционной системы, выделенны
% Поскольку веб-портал не имеет выделенного аппаратного обеспечения, он не имеет прямых аппаратных интерфейсов.

\subsection{Программные интерфейсы}
Среди программных интерфейсов целой системы должны присутствовать \begin{itemize}
    \item интерфейс доступа сервера к произвольным данным в хранилище для чтения или изменения, а так же интерфейс для создания новых записей,
    \item интерфейс запроса клиентом у серверана получение данных из хранилища,
    \item интерфейс запроса клиентом у сервера изменения или создания данных в хранилище.
    \item интерфейс проверки авторизации и аутентификации пользователя
    \item и интерфейс выполнения авторизации и/или аутентификации пользователя.
    % \item \textit{[И Т.Д...]}

    %\item интерфейс получения данных любого пользователя,
    %\item интерфейс получения содержимого любого новостного материала,
    %\item интерфейс ...

\end{itemize}

%Там еще в аппаратных коммент оставил
%ок, попозже гляну
%в системе как комплекс ПО должны быть ^^^ интерфейсы imho
%Это разве не пользовательские?
%Они заложены тут VVV же
%в принципе можно написать общими словами, только нужно потом не забыть это конкретно расписать в функциональные возможности.


%Связь между базой данных и веб-порталом состоит из операций, касающихся как чтения, так и изменения данных.

%Дарья сказала опустить
%\subsection{Коммуникационные интерфейсы}
%Связь между различными частями системы важна, поскольку они зависят друг от друга. Однако то, каким образом достигается связь, не имеет значения для системы и, следовательно, обрабатывается используемыми операционными системами как для клиентского приложения, так и для серверного приложения.

%%%%
% \textit{[This should be а detailed description of all inputs into and outputs from the software system. It should complement the interface descriptions in 5.2 and should not repeat information there.
% It should include both content and format as follows:\\
% а) Name of item\\
% Ь) Description of purpose\\
% с) Source of input or destination of output\\
% d) Valid range, ассигасу апШог tolerance\\
% е) Units of measure\\
% f) Timing\\
% g) Relationships to other inputs/outputs\\
% h) Screen formats/organization\\
% i) Window formats/organization\\
% ј) Data formats\\
% К) Command formats\\
% l) End messages\\
% ]}

\section{Функциональные характеристики системы}
В этом разделе приводится подробное описание всех интерфейсов ввода и вывода. Все интерфейсы разделены на области действия: аппаратные, программного обеспечения, коммуникационные между копмонентами системы и пользовательские. В этом разделе документа так же представлен прототип пользовательского интерфейса веб-портала.

\subsection{Пользовательские интерфейсы}
\begin{enumerate}
    \item <<Главная страница>>\\
    На данной странице система должна предоставлять пользователю набор последних актуальных новостей, а также прогноз погоды, курс валют и доступ к поиску по новостям.
    \item <<Страница с фильтрами>>\\
    На данной странице система должна предоставлять пользователю список фильтров по новостям, а также найденные новости по заданному фильтру.
    \item <<Страница с новостью>>\\
    На данной странице система должна предоставлять пользователю новостной материал по данной ссылке.
    \item <<Страница авторизации>>\\
    На данной странице система должна предоставлять пользователю форму для авторизации пользователя.
    \item <<Страница с панелью управления>>\\
    На данной странице система должна предоставлять авторизованному пользователю визуальный интерфейс для выполнения дейтвий соответствующих роли пользователя.
\end{enumerate}
%\textit{[USER INTERFACES --- UI]}

\subsection{Аппаратные интерфейсы}
Среди аппаратных интерфейсов можно выделить интерфейс доступа в глобальную сеть. 

На стороне клиента должно присутсвовать соедениене с глобальной сетью Интернет по средствам доступных подключений (сотовая связь, проводное подключение, беспроводная сеть Wi-Fi и пр.) для удобства пользователя. Клиентское приложение может отправлять пакеты только на адрес сервера системы и адреса серверов для получения дополнительного ресурса и получать пакеты от них по сети.
%Это хоть и важный интерфейс, но он не заложен  в разработку же
%Мне казалось, что здесь про интерфейсы, которые мы разрабатываем
%думаю стоит оставить, если что потом уберем. Разрабоатывая систему мы в полне можем столкнуться с таким интефейсом, там всякие CORS тоже ведь сюда, кажется

Серверная часть также должна иметь доступ к сети Интернет и локальной сети по средствам проводного подключения для снижения временных задержек на обмен информацией. Серверное приложение может получать пакеты клиентских запросов с любого адреса и отправлять соответствующие пакеты ответов на те же адреса. Так же серверное приложение может отправлять запросы на адрес СУБД, если она физически находится на другом сервере подключенном к сети.

% Поскольку система функционирует на уровне операционной системы, выделенны
% Поскольку веб-портал не имеет выделенного аппаратного обеспечения, он не имеет прямых аппаратных интерфейсов.

\subsection{Программные интерфейсы}
Среди программных интерфейсов целой системы должны присутствовать \begin{itemize}
    \item интерфейс доступа сервера к произвольным данным в хранилище для чтения или изменения, а так же интерфейс для создания новых записей,
    \item интерфейс запроса клиентом у серверана получение данных из хранилища,
    \item интерфейс запроса клиентом у сервера изменения или создания данных в хранилище.
    \item интерфейс проверки авторизации и аутентификации пользователя
    \item и интерфейс выполнения авторизации и/или аутентификации пользователя.
    % \item \textit{[И Т.Д...]}

    %\item интерфейс получения данных любого пользователя,
    %\item интерфейс получения содержимого любого новостного материала,
    %\item интерфейс ...

\end{itemize}

%Там еще в аппаратных коммент оставил
%ок, попозже гляну
%в системе как комплекс ПО должны быть ^^^ интерфейсы imho
%Это разве не пользовательские?
%Они заложены тут VVV же
%в принципе можно написать общими словами, только нужно потом не забыть это конкретно расписать в функциональные возможности.


%Связь между базой данных и веб-порталом состоит из операций, касающихся как чтения, так и изменения данных.

%Дарья сказала опустить
%\subsection{Коммуникационные интерфейсы}
%Связь между различными частями системы важна, поскольку они зависят друг от друга. Однако то, каким образом достигается связь, не имеет значения для системы и, следовательно, обрабатывается используемыми операционными системами как для клиентского приложения, так и для серверного приложения.

%%%%
% \textit{[This should be а detailed description of all inputs into and outputs from the software system. It should complement the interface descriptions in 5.2 and should not repeat information there.
% It should include both content and format as follows:\\
% а) Name of item\\
% Ь) Description of purpose\\
% с) Source of input or destination of output\\
% d) Valid range, ассигасу апШог tolerance\\
% е) Units of measure\\
% f) Timing\\
% g) Relationships to other inputs/outputs\\
% h) Screen formats/organization\\
% i) Window formats/organization\\
% ј) Data formats\\
% К) Command formats\\
% l) End messages\\
% ]}

\section{Требования к производительности}
в этой секции представлены технические требования к производительности, определенные на основании ранее представленных зависимостей и тревований.

повседневной нагрузкой будем считать обращение пятой части всех жителей региона амурская область к сервису за сутки. это по приблизительным рассчетам составляет $$700\,000\,/\,5\, /\,24\,/\,60\,\approx 100\;\text{запросов/час}.$$

пиковой нагрузкой на систему считается превышение обычной нагрузки в 2 раза.
%здесь видимо таблицу с атрибутами тогда вводить
%в примере первом там через функцию как у нас
%да зачем?

\begin{enumerate}
    \QualityReq{Повседневная нагрузка}{QR1}{Система должна выдерживать повседневную нагрузку неограниченное время.}
    
    \QualityReq{Время безотказной работы}{QR2}{Система должна находиться непрерывно в работоспособном состоянии не менее 95\% времени в течение года.}
    
    \QualityReq{Максимальное время ответа}{QR3}{Система должна давать давать ответ на запрос не дольше чем за 10 с. в условиях нахождения пользователя в амурской области и скорости интернета не менее 250 кбит/с.}
    
    \QualityReq{Время выбора и сортировки}{QR4}{База данных должна выполнять запрос на выбор и сортировку по определенным критериям за время не больше установленного времени ответа на запрос.}
    
    \QualityReq{Резервное копирование}{QR5}{База данных должна давать возможность делать резервные копии в любое время.}
    
    \QualityReq{Пиковая нагрузка}{QR6}{Система должна выдерживать круглосуточно пиковую нагрузку без значительных просадок в производительности в течение пары дней.}
  %  \item [И ТД...]
\end{enumerate}

% \textit{[5.3.3 in IEEE] This subsection should specify both the static and the dynamic numerical requirements placed оп the soft-
% ware or оп human interaction with the software as а whole. Static numerical requirements тау include:
% \\
% а) The ofterminals to be supported\\
% Ь) Пе number of simultaneous users to be supported\\
% с) Amount and of information to бе handled
% \\
% Static numerica1 requirements аге sometimes identifed under а separate section entit1ed capacity.
% \\
% Dynamic numerical requirements тау include, for example, the numbers of transactions and tasks and the
% amount of data to pr(xessed within certain time periods for both normal and реак workload conditions.\\
% All of these requirements should be stated in measurable terms.
% \\
% For ехатр1е,\\
% ~~~95\% ofthe transactions shall be processed in less than 1 s
% \\
% rather than,\\
% ~~~Ап operator shall пот have го waitfor the transaction го complete.
% \\
% NOTE—Numerical limits applied to опе specifc function ате normally specifed as part of the subparagraph
% description of that function.
% }

\section{Логические требования к СУБД}
В этой секции представлены требования к СУБД на основе предполагаемых сценариев использования сервисов.

\begin{enumerate}
    \QualityReq{Редактирование данных}{QR7}{База данных должна предоставлять возможность редактировать записи.}
    \QualityReq{Авторство записи}{QR8}{Каждая сущность новостного материала должна иметь ссылку на сущность своего создателя.}
    \QualityReq{Безграничный объём}{QR9}{База данных не должна иметь ограничение на максимальных размер хранимых данных, в частности на исходный код разметки новостных материалов.}
    \QualityReq{Безопасность пользовательских данных}{QR10}{Пароли пользователей должны храниться в базе данных в зашифрованном виде.}
    \QualityReq{Качество СУБД}{QR11}{СУБД должна быть хорошей, плохой быть не должна.}
    \QualityReq{Расширяемость СУБД}{QR12}{База данных должна иметь продуманную архитектуру для дальнейшего раcширения и сохранения хорошей производительности.}
\end{enumerate}

% \textit{[5.3.4 in IEEE] This should specify the 10gical requirements for апу infonnation that is to be placed into а database. This
% тау include:
% \\
% а) TYpes of information used Ьу various functions\\
% Ь) Frequency of use\\
% с) Accessing capabilities\\
% d) Data entities and their relationships\\
% е) Integrity constraints\\
% f) Data retention requirements
% }

\section{Требования к дизайну системы}
В этой секции описаны требования к подходу к реализации системы, методологии разработки, дизайну архитектуры системы.

\begin{enumerate}
    \QualityReq{Развитие системы}{QR13}{Система должна позволять дальнейшее развитие и расщирение функционала.}
    \QualityReq{Отсутствие привязки к платформе}{QR14}{Система не должна быть привязана к определенной операционной системе.}
    \QualityReq{Одинаковая функциональность}{QR15}{Клиентское приложение должно иметь одинаковую функциональность на любой возможной платформе пользователя.}
    \QualityReq{Адаптивный интерфейс}{QR16}{Пользовательский интерфейс гостей должен быть адаптирован как под десктопные, так и под мобильные устройства.}
    \QualityReq{Лицензирование}{QR17}{Система должна состоять из компонентов, лицензия которых допускает коммерческое использование, конечный выбор должен быть оговорен с заказчиком продукта.}
\end{enumerate}


% \textit{[5.3.5 in IEEE] This should specify design constraints that сап be imposed by other standards, hardware limitations, etc.}

% \subsection{Standart compliance}
% This subsection should specify the requirements derived from existing standards or regulations. They
% тау include

% а) Report format
% Ь) Data naming
% с) Accounting procedures
% d) Audit tracing

% For ехатр1е, this could specify the requirement for software to trace processing activity. Such traces ате
% needed for some applications to meet minimum regulatory ог fnancial standards. Ап audit trace requirement
% тау, for ехатр1е, state that a11 changes to а payroll database must be recorded in а trace fle with before and
% after values.

% \section{Аспекты системы}
% В этом разделе приводится подробное описание всех интерфейсов ввода и вывода. Все интерфейсы разделены на области действия: аппаратные, программного обеспечения, коммуникационные между копмонентами системы и пользовательские. В этом разделе документа так же представлен прототип пользовательского интерфейса веб-портала.

\subsection{Пользовательские интерфейсы}
\begin{enumerate}
    \item <<Главная страница>>\\
    На данной странице система должна предоставлять пользователю набор последних актуальных новостей, а также прогноз погоды, курс валют и доступ к поиску по новостям.
    \item <<Страница с фильтрами>>\\
    На данной странице система должна предоставлять пользователю список фильтров по новостям, а также найденные новости по заданному фильтру.
    \item <<Страница с новостью>>\\
    На данной странице система должна предоставлять пользователю новостной материал по данной ссылке.
    \item <<Страница авторизации>>\\
    На данной странице система должна предоставлять пользователю форму для авторизации пользователя.
    \item <<Страница с панелью управления>>\\
    На данной странице система должна предоставлять авторизованному пользователю визуальный интерфейс для выполнения дейтвий соответствующих роли пользователя.
\end{enumerate}
%\textit{[USER INTERFACES --- UI]}

\subsection{Аппаратные интерфейсы}
Среди аппаратных интерфейсов можно выделить интерфейс доступа в глобальную сеть. 

На стороне клиента должно присутсвовать соедениене с глобальной сетью Интернет по средствам доступных подключений (сотовая связь, проводное подключение, беспроводная сеть Wi-Fi и пр.) для удобства пользователя. Клиентское приложение может отправлять пакеты только на адрес сервера системы и адреса серверов для получения дополнительного ресурса и получать пакеты от них по сети.
%Это хоть и важный интерфейс, но он не заложен  в разработку же
%Мне казалось, что здесь про интерфейсы, которые мы разрабатываем
%думаю стоит оставить, если что потом уберем. Разрабоатывая систему мы в полне можем столкнуться с таким интефейсом, там всякие CORS тоже ведь сюда, кажется

Серверная часть также должна иметь доступ к сети Интернет и локальной сети по средствам проводного подключения для снижения временных задержек на обмен информацией. Серверное приложение может получать пакеты клиентских запросов с любого адреса и отправлять соответствующие пакеты ответов на те же адреса. Так же серверное приложение может отправлять запросы на адрес СУБД, если она физически находится на другом сервере подключенном к сети.

% Поскольку система функционирует на уровне операционной системы, выделенны
% Поскольку веб-портал не имеет выделенного аппаратного обеспечения, он не имеет прямых аппаратных интерфейсов.

\subsection{Программные интерфейсы}
Среди программных интерфейсов целой системы должны присутствовать \begin{itemize}
    \item интерфейс доступа сервера к произвольным данным в хранилище для чтения или изменения, а так же интерфейс для создания новых записей,
    \item интерфейс запроса клиентом у серверана получение данных из хранилища,
    \item интерфейс запроса клиентом у сервера изменения или создания данных в хранилище.
    \item интерфейс проверки авторизации и аутентификации пользователя
    \item и интерфейс выполнения авторизации и/или аутентификации пользователя.
    % \item \textit{[И Т.Д...]}

    %\item интерфейс получения данных любого пользователя,
    %\item интерфейс получения содержимого любого новостного материала,
    %\item интерфейс ...

\end{itemize}

%Там еще в аппаратных коммент оставил
%ок, попозже гляну
%в системе как комплекс ПО должны быть ^^^ интерфейсы imho
%Это разве не пользовательские?
%Они заложены тут VVV же
%в принципе можно написать общими словами, только нужно потом не забыть это конкретно расписать в функциональные возможности.


%Связь между базой данных и веб-порталом состоит из операций, касающихся как чтения, так и изменения данных.

%Дарья сказала опустить
%\subsection{Коммуникационные интерфейсы}
%Связь между различными частями системы важна, поскольку они зависят друг от друга. Однако то, каким образом достигается связь, не имеет значения для системы и, следовательно, обрабатывается используемыми операционными системами как для клиентского приложения, так и для серверного приложения.

%%%%
% \textit{[This should be а detailed description of all inputs into and outputs from the software system. It should complement the interface descriptions in 5.2 and should not repeat information there.
% It should include both content and format as follows:\\
% а) Name of item\\
% Ь) Description of purpose\\
% с) Source of input or destination of output\\
% d) Valid range, ассигасу апШог tolerance\\
% е) Units of measure\\
% f) Timing\\
% g) Relationships to other inputs/outputs\\
% h) Screen formats/organization\\
% i) Window formats/organization\\
% ј) Data formats\\
% К) Command formats\\
% l) End messages\\
% ]}


%%%%
% This section of the SRS should contain all the software Шrequirements to а level of detail suffcient to enable
% designers to design а system to satisfy those requirements, and testers to test that the system satisfes those
% requirements. Throughout this section, every stated requirement should Ь-е extemally perceivable by users,
% operators, от other external systems. These requirements should include at а minimum а description of every
% input (stimulus) into the system, every output (response) from the system and all functions performed by the
% system in response to ап input от in support of ап output. As this is often the largest and most important part
% of the SRS, the following principles арр1у:
% 
% а) Specifc requirements should be stated in conformance with all the characteristics described in 4.3 0f this recommended practice.\\
% b) Specifc  requirements should be cross-referenced to earlier documents relate.\\
% с) All requirements should be uniquely identifable.\\
% d) Careful attention should given to organizing the requirements to maximize readability.\\
