В этом разделе приводится подробное описание всех интерфейсов ввода и вывода. Все интерфейсы разделены на области действия: аппаратные, программного обеспечения, коммуникационные между копмонентами системы и пользовательские. В этом разделе документа так же представлен прототип пользовательского интерфейса веб-портала.

\subsection{Пользовательские интерфейсы}
\begin{enumerate}
    \item <<Главная страница>>\\
    На данной странице система должна предоставлять пользователю набор последних актуальных новостей, а также прогноз погоды, курс валют и доступ к поиску по новостям.
    \item <<Страница с фильтрами>>\\
    На данной странице система должна предоставлять пользователю список фильтров по новостям, а также найденные новости по заданному фильтру.
    \item <<Страница с новостью>>\\
    На данной странице система должна предоставлять пользователю новостной материал по данной ссылке.
    \item <<Страница авторизации>>\\
    На данной странице система должна предоставлять пользователю форму для авторизации пользователя.
    \item <<Страница с панелью управления>>\\
    На данной странице система должна предоставлять авторизованному пользователю визуальный интерфейс для выполнения дейтвий соответствующих роли пользователя.
\end{enumerate}
%\textit{[USER INTERFACES --- UI]}

\subsection{Аппаратные интерфейсы}
Среди аппаратных интерфейсов можно выделить интерфейс доступа в глобальную сеть. 

На стороне клиента должно присутсвовать соедениене с глобальной сетью Интернет по средствам доступных подключений (сотовая связь, проводное подключение, беспроводная сеть Wi-Fi и пр.) для удобства пользователя. Клиентское приложение может отправлять пакеты только на адрес сервера системы и адреса серверов для получения дополнительного ресурса и получать пакеты от них по сети.
%Это хоть и важный интерфейс, но он не заложен  в разработку же
%Мне казалось, что здесь про интерфейсы, которые мы разрабатываем
%думаю стоит оставить, если что потом уберем. Разрабоатывая систему мы в полне можем столкнуться с таким интефейсом, там всякие CORS тоже ведь сюда, кажется

Серверная часть также должна иметь доступ к сети Интернет и локальной сети по средствам проводного подключения для снижения временных задержек на обмен информацией. Серверное приложение может получать пакеты клиентских запросов с любого адреса и отправлять соответствующие пакеты ответов на те же адреса. Так же серверное приложение может отправлять запросы на адрес СУБД, если она физически находится на другом сервере подключенном к сети.

% Поскольку система функционирует на уровне операционной системы, выделенны
% Поскольку веб-портал не имеет выделенного аппаратного обеспечения, он не имеет прямых аппаратных интерфейсов.

\subsection{Программные интерфейсы}
Среди программных интерфейсов целой системы должны присутствовать \begin{itemize}
    \item интерфейс доступа сервера к произвольным данным в хранилище для чтения или изменения, а так же интерфейс для создания новых записей,
    \item интерфейс запроса клиентом у серверана получение данных из хранилища,
    \item интерфейс запроса клиентом у сервера изменения или создания данных в хранилище.
    \item интерфейс проверки авторизации и аутентификации пользователя
    \item и интерфейс выполнения авторизации и/или аутентификации пользователя.
    % \item \textit{[И Т.Д...]}

    %\item интерфейс получения данных любого пользователя,
    %\item интерфейс получения содержимого любого новостного материала,
    %\item интерфейс ...

\end{itemize}

%Там еще в аппаратных коммент оставил
%ок, попозже гляну
%в системе как комплекс ПО должны быть ^^^ интерфейсы imho
%Это разве не пользовательские?
%Они заложены тут VVV же
%в принципе можно написать общими словами, только нужно потом не забыть это конкретно расписать в функциональные возможности.


%Связь между базой данных и веб-порталом состоит из операций, касающихся как чтения, так и изменения данных.

%Дарья сказала опустить
%\subsection{Коммуникационные интерфейсы}
%Связь между различными частями системы важна, поскольку они зависят друг от друга. Однако то, каким образом достигается связь, не имеет значения для системы и, следовательно, обрабатывается используемыми операционными системами как для клиентского приложения, так и для серверного приложения.

%%%%
% \textit{[This should be а detailed description of all inputs into and outputs from the software system. It should complement the interface descriptions in 5.2 and should not repeat information there.
% It should include both content and format as follows:\\
% а) Name of item\\
% Ь) Description of purpose\\
% с) Source of input or destination of output\\
% d) Valid range, ассигасу апШог tolerance\\
% е) Units of measure\\
% f) Timing\\
% g) Relationships to other inputs/outputs\\
% h) Screen formats/organization\\
% i) Window formats/organization\\
% ј) Data formats\\
% К) Command formats\\
% l) End messages\\
% ]}