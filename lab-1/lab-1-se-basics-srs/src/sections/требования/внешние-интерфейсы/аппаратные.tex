Среди аппаратных интерфейсов можно выделить интерфейс доступа в глобальную сеть. 

На стороне клиента должно присутсвовать соедениене с глобальной сетью Интернет по средствам доступных подключений (сотовая связь, проводное подключение, беспроводная сеть Wi-Fi и пр.) для удобства пользователя. Клиентское приложение может отправлять пакеты только на адрес сервера системы и адреса серверов для получения дополнительного ресурса и получать пакеты от них по сети.
%Это хоть и важный интерфейс, но он не заложен  в разработку же
%Мне казалось, что здесь про интерфейсы, которые мы разрабатываем
%думаю стоит оставить, если что потом уберем. Разрабоатывая систему мы в полне можем столкнуться с таким интефейсом, там всякие CORS тоже ведь сюда, кажется

Серверная часть также должна иметь доступ к сети Интернет и локальной сети по средствам проводного подключения для снижения временных задержек на обмен информацией. Серверное приложение может получать пакеты клиентских запросов с любого адреса и отправлять соответствующие пакеты ответов на те же адреса. Так же серверное приложение может отправлять запросы на адрес СУБД, если она физически находится на другом сервере подключенном к сети.

% Поскольку система функционирует на уровне операционной системы, выделенны
% Поскольку веб-портал не имеет выделенного аппаратного обеспечения, он не имеет прямых аппаратных интерфейсов.